% Capítulo 1

\chapter{El problema} % Título Principal del Capítulo

\label{Capitulo1} % Para hacer referenciar este capítudo use \ref{Capitulo1}

%----------------------------------------------------------------------------------------
%	SECTION 1
%----------------------------------------------------------------------------------------

\section{PLANTEAMIENTO DEL PROBLEMA}

El estudio de estructuras radiantes amerita de la aplicación rigurosa de las ecuaciones de Maxwell en todos los elementos que la constituyen físicamente; de esta forma, es posible determinar su comportamiento en condiciones concretas y caracterizarlas mediante el cálculo de parámetros electromagnéticos; sin embargo, el desarrollo matemático inherente a este análisis no siempre es factible mediante técnicas convencionales; debido a esto, el estudiante se encuentra en la necesidad de utilizar recursos computacionales que le permitan caracterizar cualitativa y cuantitativamente las diferentes estructuras que requiera examinar. Aunado a esto, la mayoría de herramientas disponibles para tal fin, por lo general, no proporcionan una interfaz didáctica que facilite su uso en el ámbito educacional; además, constituyen, en su mayoría, software propietarios cuyo uso implica la adquisición de licencias para  acceder a todas sus funcionalidades, implicando una inversión monetaria por parte del usuario.\\

En consecuencia, el estudiante debe recurrir a aplicaciones de licencia pública general (GNU) que presentan la limitación de no permitir el diseño y simulación de antenas muy complejas, obstaculizando así el trabajo experimental y forzando al estudiante a recurrir a varias aplicaciones que, en su conjunto, le ayuden a resolver el problema presentado. A partir de esto, es posible afirmar que en la actualidad no está disponible ningún software de código abierto cuyas características y funcionalidades estén adaptadas para su uso específico en las prácticas del laboratorio de Antenas y Propagación de la escuela de Telecomunicaciones; por ende, el estudiante debe emplear numerosas aplicaciones de distintas características para llevar a cabo los cálculos matemáticos inherentes al fenómenos electromagnético específico bajo análisis, dificultando así un aprendizaje enfocado en el concepto físico desarrollado en la práctica experimental para poder comprender de forma clara el fenómeno representado mediante la simulación. 

En atención a estas necesidades planteadas y con la finalidad de facilitar un mejor aprendizaje de la asignatura, se propone el diseño de una herramienta computacional didáctica e interactiva enfocada en las prácticas de laboratorio y con la misma robustez de un software de propietario, el cual será denominado UCNEC.


%----------------------------------------------------------------------------------------
%	SECTION 2
%----------------------------------------------------------------------------------------
\newpage
\section{JUSTIFICACIÓN DE LA INVESTIGACIÓN}

Partiendo de la problemática planteada, se decide diseñar una herramienta virtual que proporcione un entorno adecuado para el modelado y simulación de estructuras radiantes; para tal fin, se resuelve la utilización del lenguaje JAVA toda vez que ofrece las ventajas de la codificación imperativa dentro del paradigma de la programación orientada a objetos; en este mismo orden de ideas, se trata de una alternativa de código abierto, multiplataforma, de uso extendido y con una dinámica comunidad de desarrolladores que ofrecen, de forma continua y abierta, una importante cantidad de recursos útiles para la implementación de operaciones matemáticas complejas y visualización de datos científicos; del mismo modo, el estudio de este lenguaje de programación forma parte del pensum de la escuela de ingeniería de telecomunicaciones a través de la cátedra de computación avanzada, constituyendo una oportunidad para aplicar los conocimientos adquiridos por medio de la misma.\\
El desarrollo de este programa permitirá eludir el uso de software propietario, el cual supone una importante inversión monetaria por concepto de licencias; además, ya que integrará en una sola entidad los diferentes \textit{software} utilizados en la cátedra de Antenas y Propagación, se podrá obviar el tedioso procedimiento de adquirir e instalar los numerosos recursos, hasta ahora requeridos por la materia, para el desarrollo de sus respectivas prácticas de laboratorio.\\
Se plantea que la interacción con la aplicación propuesta sea lo suficientemente fluída y didáctica para que los esfuerzos del usuario se enfoquen en una comprensión efectiva de los contenidos desarrollados por la cátedra y menos hacia el uso de la herramienta; de este modo, se busca incrementar, a través de esta propuesta, el desempeño académico de los estudiantes en la asignatura, suscitando, por añadidura, una integración natural de los conocimientos adquiridos con anterioridad referentes a los fenómenos electromagnéticos y de radiación; para lograrlo, el diseño propone más que un ensamblaje de distintos programas, un ecosistema en el cual para cada problema planteado dentro de los objetivos de una práctica de laboratorio, existan diferentes medios desde los cuales pueda ser abordado.\\
%Cabe destacar que el uso de buena parte de las aplicaciones de modelado y simulación de antenas está supeditado a la ejecución de un sistema operativo específico, el cual puede no ser el utilizado por algunos estudiantes, traduciéndose en un esfuerzo e inversión de tiempo adicional por parte de estos para adaptarse a los requerimientos de unas herramientas que no exhiben suficiente flexibilidad para con los usuarios; en este sentido, surge la necesidad de desarrollar una opción que posibilite enfocar la mayor parte de los esfuerzos de los estudiantes en la consecución de los objetivos planteados por la materia y menos en la adaptación de sus espacios de trabajo a determinadas condiciones arbitrarias.\\


%----------------------------------------------------------------------------------------
%	SECTION 3
%---------------------------------------------------------------------------------------

\section{OBJETIVOS}


\subsection{Objetivo General}
Diseñar una aplicación didáctica basada en el código NEC2, para la simulación y estudio del comportamiento de sistemas radiantes.

\subsection{Objetivos Específicos}

\begin{itemize}
\item Identificar la estructura de los archivos generados a partir delcnúcleo de NEC2 para adaptarlo a un sistema de procesamiento de datos construído a partir del lenguaje de programación Java.\\

\item  Efectuar la conversión de los parámetros geométricos y electromagnéticos de simulación de antenas en comandos interpretables por el código NEC2 para la implementación de estudios de su comportamiento según los criterios introducidos por el usuario del entorno virtual.\\

\item Examinar las prácticas de laboratorio diseñadas por la cátedra de Antenas y Propagación para su integración en el entorno virtual desarrollado.\\

\item Diseñar una interfaz de usuario para el modelaje y simulación de comportamientode antenas, implementando las funcionalidades ofrecidas por el código NEC2.



%----------------------------------------------------------------------------------------
%	SECTION 4
%---------------------------------------------------------------------------------------

\section{ALCANCE}
